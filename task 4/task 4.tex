\documentclass{article}
\usepackage[english,russian]{babel}
\usepackage[left=1cm,right=2cm,top=2cm,bottom=2cm,footskip=1cm]{geometry}
\usepackage{ amssymb }
\usepackage{amsmath} 
\usepackage{listings}
\usepackage{xcolor}
\lstset { %
	language=Python,
	backgroundcolor=\color{black!5}, % set backgroundcolor
	basicstyle=\footnotesize,% basic font setting
}

\title{Домашняя работа 4}
\author{Пасечник Даша}
\date{на 14.03.2019}

\usepackage{natbib}
\usepackage{graphicx}

\begin{document}

\maketitle

\subsection*{Задача 1}
\textit{Определите, являются ли задачи выполнимости и тавтологичности булевой формулы в ДНФ $\mathcal{P}$, $ \mathcal{NP}c$ или $co\mathcal{NP}c$.}

\subsubsection*{Выполнимые ДНФ}
Построим характеристическую функцию для $L$, т.е покажем, что $L \in \mathcal{P}$, значит $L \notin \mathcal{NP}c$, $L \notin co\mathcal{NP}c$.\\
Заметим, что: 
\begin{enumerate}
	\item ДНФ выполнима, если в ней выполним хотя бы 1 коньюнкт;
	\item коньюнкт выполним, если в нем не встречается одновременно переменная и ее отрицание.
\end{enumerate}
Тогда $\chi_L$ делает следующее: идет по формуле и для каждого конъюнкта проверяет присутствует ли в нем переменная и ее отрицание одновременно. Если нашелся конъюнкт, для которого это не выполняется, значит ДНФ выполнима, $\chi_L$ возвращает 1. Иначе, если во всех конъюнктах содержатся и переменная, и ее отрицание одновременно, $\chi_L$ возвращает 0. Конъюнктов конечное число, переменных в конъюнкте -- тоже, значит $\chi_L$ полиномиальна.

\subsubsection*{Тавтологичные ДНФ}
Рассмотрим дополнение к $L$ - язык нетавтологичных ДНФ, т.е. $L^* = \{\phi | \exists x : \phi(x) = 0\}$.\\
Сведем $SAT$ - язык выполнимых КНФ к нему. Рассмотрим $f(\phi) = \neg \phi$. $f$ строит отрицание $\phi$ следующим образом:
\begin{enumerate}
	\item Каждая конъюнкция заменяется на дизъюнкцию.
	\item Каждая дизъюнкция заменяется на конъюнкцию.
	\item Каждая переменная заменяется на ее отрицание.
	\item Каждое отрицание переменной заменяется на переменную.
\end{enumerate} 
Это в точности применение правила де Моргана к отрицанию КНФ формы. Заметим, что после такого преобразование $f(\phi)$ окажется ДНФ. Если $\phi \in SAT$, то $\exists x: \phi(x)=1$, тогда $\neg \phi(x) = 0$. Т.е. $\neg \phi$ нетавтологична и ДНФ, значит $f(\phi) \in L^*$. Аналогично, если $\phi \notin SAT$, то $\forall x : \phi(x) = 0$, тогда $\forall x : \neg \phi(x) = 1$, т.е. $f(\phi)$ по-прежнему ДНФ и тавтологична, значит $f(\phi) \notin L^*$. $f$ полиномиальна, т.к. идем 1 раз циклом по длине входа, на каждом шаге операции занимают $O(1)$.\\
Показали, что $SAT \leq_p L^*$. Значит $L^* \in \mathcal{NP}c$, $L \in co\mathcal{NP}c$. К слову, именно это и нужно было в 3 задаче прошлого дз.
\subsection*{Задача 2}
\textit{Под $3SAT$ обычно имеется в виду множество выполнимых КНФ с не более чем тремя переменными в каждом дизъюнкте. Покажите, что это полиномиально равнозначно $EXACTLY3SAT$, то есть с ровно тремя переменные в дизъюнкте.}\\
\subsubsection*{$3SAT \leq_p EXACTLY3SAT$}
Рассмотрим каждый дизъюнкт в $3SAT$:
\begin{itemize}
	\item Если в дизъюнкте 3 литерала, то $f$ оставляет его как есть.
	\item Если в дизъюнкте 2 литерала, т.е. он имеет вид $(x \vee y)$, то  $f$ заменяет его на $(x \vee y \vee z) \wedge (x \vee y \vee \neg z)$. Понятно, что замена выполняется тогда и только тогда, когда выполняется исходная формула, при любых $z$.
	\item Если в дизъюнкте 1 литерал, т.е. он имеет вид $(x)$, то $f$ заменяет его на $(x \vee y) \wedge (x \vee \neg y)$ и тем сводит задачу к предыдущей.
\end{itemize}
Понятно, что $f$ полиномиальна и $f(\phi)$ это $\phi$, приведенная к КНФ. Значит $\phi \in 3SAT \Leftrightarrow f(\phi) \in EXACTLY3SAT$, это и есть определение полиномиальной сводимости. 
\subsubsection*{$EXACTLY3SAT \leq_p 3SAT$}
Рассмотрим каждый дизъюнкт в $EXACTLY3SAT$:
\begin{itemize}
	\item Если в дизъюнкте 3 литерала, то $f$ оставляет его как есть.
	\item Если в дизъюнкте 2 литерала, т.е. он имеет вид $(x \vee y)$, то  $f$ заменяет его на $(x \vee y \vee z \vee p)$. Понятно что такое преобразование неэквивалентно, но это и неважно.
	\item Если в дизъюнкте 1 литерал, т.е. он имеет вид $(x)$, то $f$ заменяет его на $(x \vee y \vee z \vee p)$.
\end{itemize}
Понятно, что $f$ полиномиальна. Если $\phi \in EXACTLY3SAT$, то $f(\phi) = \phi$, а значит $f(\phi) \in 3SAT$ по определению. Если $\phi \notin EXACTLY3SAT$, то либо в $f$ изначально есть дизъюнкты с количеством литералов большим 3, либо все дизъюнкты содержат ровно 3 литерала и $f$ невыполнима, либо изначально в $f$ есть только дизъюнкты с меньше либо равным 3 количеством литералов. В первом случае в $f(\phi)$ также окажутся дизъюнкты с количеством литералов большим 3, т.к. $f$ их не трогает, значит $f(\phi) \notin 3SAT$. Во втором случае $f(\phi) = \phi$, значит $f(\phi) \notin 3SAT$ как невыполнимая. В третьем случае $f(\phi)$ преобразует все дизъюнкты с количеством литералов меньшим 3 к дизъюнктам с количеством литералов равным 4, значит $f(\phi) \notin 3SAT$. Искомая сводимость доказана.\\
\\
$3SAT =_p EXACTLY3SAT$
\subsection*{Задача 3}
\textit{Докажите, что задача $VERTEX$-$COVER \in \mathcal{NP}c$}
\subsubsection*{$VERTEX$-$COVER \in \mathcal{NP}$}
Предикат получает на вход набор вершин $V$ размера $k$ и проверяет для каждого ребра есть ли хотя бы 1 из его вершин в $V$. Ребер конечное число, значит предикат полиномиален.
\subsubsection*{$VERTEX$-$COVER \in \mathcal{NP}c$}
Построим сводимость $CLIQUE \leq_p VERTEX-COVER$.\\
Рассмотрим $f((G, k)) = (G', k')$, где $k' = n - k$, $n$ - количество вершин в графе, $G'$ - дополнение к $G$, т.е. есть ребра, которых нет в $G$, нет ребер, которые есть в $G$. $f$ один раз проходит по матрице смежности, значит полиномиальна.\\
Пусть $(G, k) \in CLIQUE$. Обозначим за $A$ множество вершин, образующих клику, за $B$ - множество оставшихся вершин. Заметим, что у каждого ребра из $G'$ хотя бы 1 вершина лежит в $B$. Значит вершины $B$ образуют вершинное покрытие и его размер $n - k = k'$. Т.е. $f((G, k)) \in VERTEX-COVER$.\\
Пусть $(G, k) \notin CLIQUE$. Пусть в $G'$, есть вершинное покрытие размера $k'$, рассмотрим множество вершин, не входящих в него - $A$. Все вершины из $A$ в $G'$ не связаны между собой, но тогда в $G$ они образовывали клику размера $n - k' = k$. Противоречие. Значит $f((G, k)) \notin VERTEX-COVER$.
\subsection*{Задача 4}
\textit{Докажите, что задача ПРОТЫКАЮЩЕЕ-МНОЖЕСТВО $\in \mathcal{NP}c$.}
\subsubsection*{$L \in \mathcal{NP}$}
Предикат получает на вход множество $A$ из $k$ элементов и проверяет пересечение со всеми множествами из $A_{\phi}$. Если все пересечения непусты - выдает 1. Множеств в $A_{\phi}$ конечное число, и каждое конечно, значит предикат полиномиален. 
\subsubsection*{$L \in \mathcal{NP}c$}
Построим сводимость $CNFSAT \leq_p L$.
Пусть $\phi((x_1, \cdots , x_n))$ КНФ. Построим по КНФ семейство подмножеств $A_{\phi}$ базового множества $\{x_1, \cdots, x_n, \neg x_1, \cdots , \neg x_n\}$. Во-первых, включим в $A_{\phi}$ $n$ подмножеств вида $A_i = \{x_i, \neg x_i\}, \quad i = 1, \cdots, n$. Во-вторых, для каждого дизъюнкта $C$, входящего в $\phi$, добавим к $A_{\phi}$ подмножество $A_C$, состоящее из всех входящих в $C$ логических переменных (если в $C$ входит логическая переменная $x_i$, то включаем в $A_C$ элемент $x_i$ , а если в $C$ входит переменная $\neg x_i$, то включаем в $A_C$ элемент $\neg x_i$).\\
Пусть $\phi \in CNFSAT$. Значит $\exists x = (x_1, \cdots, x_n) : \phi(x) = 1$. Рассмотрим множество $A$: для каждого  $ i = 1, \cdots n$ в $A$ входит $x_i$, если в наборе $x$ $x_i = 1$, иначе в $A$ входит $\neg x_i$. Таким образом, в $A$ ровно $k$ элементов. Покажем, что $A$ - протыкающее множество для $A_{\phi}$ от противного. Заметим, что $A$ имеет непустое пересечение со всеми подмножествами $A_i$. Пусть существует подмножество $A_C$, построенное на основе дизъюнкта $C$, такое, что $A \cup A_C = \emptyset$. Рассмотрим логические переменные в дизъюнкте $C$ -- $\{x_{k_i}\}$. Если в С содержится $x_{k_i}$, то в $A$ содержится $\neg x_{k_i}$. Значит в наборе $x$ $x_{k_i} = 0$. Если в С содержится $\neg x_{k_i}$, то в $A$ содержится $ x_{k_i}$. Значит в наборе $x$ $x_{k_i} = 1$. Выходит что для набора $x$ дизъюнкт $C = 0$, тогда $\phi(x) \neq 1$. Противоречие. Значит все $A_C$ имеют непутое пересечение с $A$. Тогда $A$ - протыкающее множество для $A_{\phi}$ размера $n$.\\
Пусть $\phi \notin CNFSAT$. Покажем от противного, что в $A_{\phi}$ нет протыкающего множества $A$ размера $n$. Пусть есть, тогда построим набор $x$ следующим образом: если в $A$ содержится переменная, то в $x$ ее значение будет равно 1, если отрицание переменной, то 0. Очевидно, что для каждого $x_i, \quad i = 1, \cdots, x_n$ в $A$ входит либо $x_i$, либо ее отрицание, т.к. иначе нет пересечения с множеством $A_i = \{x_i, \neg x_i\}$. Рассмотрим $\phi(x)$. Для каждого дизъюнкта $A$ содержит хотя бы 1 литерал из него (т.е переменную, если в дизъюнкте содержится переменная, и отрицание переменной, если в дизъюнкте содержится отрицание переменной). Тогда в наборе $x$ значение этого литерала 1 по построению $x$. Значит значение каждого дизъюнкта 1. Значит $\phi(x) = 1$. Значит $\phi \in CNFSAT$. Противоречие. Значит в $A_{\phi}$ нет протыкающего множества $A$ размера $n$.\\
Построение $A_{\phi}$ - полиномиально.\\
*Понятно, что $CNFSAT \cup \neg CNFSAT$ это все булевы КНФ, иначе построение $A_{\phi}$ было бы некорректно.
\subsection*{Задача 5}
\textit{Покажите, что $VERTEX$-$COVER \leqslant_p SET$-$COVER$.}\\
Построим $f: \quad (G=(V, E), k) \in VERTEX$-$COVER \Leftrightarrow f((G, k)) = (U, S, k) \in SET$-$COVER$. $U$ множество элементов, $S$ семейство подмножеств $U$, $k$ - число такое, что $\exists k$ подмножеств из $S$, таких что их объединение это $U$.\\
Пусть $U = E$, В $S$ добавим $S_v = \{e \in E: e$ инцидентно $v\}$ для всех $v \in V$.\\
Пусть $(G, k) \in VERTEX$-$COVER$, т.е. $\exists X$ -- вершинное покрытие $G$ размера $k$. Тогда множества $S_v: v \in X$ образуют искомое set-cover для $U$. Его размер $k$, т.к. в $X$ $k$ вершин. Если некоторый элемент из $U$ не принадлежит никакому $S_v$ значит в $X$ нет вершины, покрывающей это ребро и $X$ - не vertex-cover. Значит $S_v$ действительно set-cover $U$ размера k.\\
Пусть $(U, S, k) \in SET$-$COVER$. Тогда $X = \{v:S_v$ входит в set-cover $U \}$ - vertex-cover размера $k$ для $G$, где $G$ такое, что $f((G, k)) = (U, S, k)$. Действительно, все элементы $U$ входят в какое-нибудь множество $S_v$, значит все ребра $G$ покрыты вершинами из $X$.\\
$f$ полиномиальна, т.к. множества вершин и ребер графа конечны.
\subsubsection*{Задача 7}
\textit{Докажите, что $\Sigma_k \cup \Pi_k \subset \Sigma_{k+1} \cap \Pi_{k+1}$.}\\
\\
Распишем по определению (б.о.о. k - четное, для нечетного аналогично):\\
$\Sigma_k = x \in A \Leftrightarrow \exists y_{1} \forall y_{2} \exists y_{3} \ldots \forall y_k : V\left(x, y_{1}, y_{2}, \ldots, y_{k}\right)=1$\\
$\Sigma_{k+1} = x \in A \Leftrightarrow \exists y_{1} \forall y_{2} \exists y_{3} \ldots \forall y_k \exists y_{k+1} V\left(x, y_{1}, y_{2}, \ldots, y_{k+1}\right)=1$\\
$\Pi_k = x \in A \Leftrightarrow \forall y_{1} \exists y_{2} \exists y_{3} \ldots \exists y_k : V\left(x, y_{1}, y_{2}, \ldots, y_{k}\right)=1$\\
$\Pi_{k+1} = x \in A \Leftrightarrow \forall y_{1} \exists y_{2} \exists y_{3} \ldots \exists y_k \forall y_{k+1} : V\left(x, y_{1}, y_{2}, \ldots, y_{k+1}\right)=1$\\
Покажем явно 4 вложения:\\
$\Sigma_k \subset \Sigma_{k+1}$. Пусть $A \in \Sigma_k$, т.е $x \in A \Leftrightarrow \exists y_{1} \forall y_{2} \exists y_{3} \ldots \forall y_k : V\left(x, y_{1}, y_{2}, \ldots, y_{k}\right)=1$. Тогда $x \in A \Leftrightarrow \exists y_{1} \forall y_{2} \exists y_{3} \ldots \forall y_k \exists y_{k+1} : V\left(x, y_{1}, y_{2}, \ldots, y_{k}\right)=1$, где $y_k+1$ фиктивная переменная, и предикат $V$ ее не использует. По определению $A \in \Sigma_{k+1}$. Значит $\Sigma_k \subset \Sigma_{k+1}$.\\
$\Sigma_k \subset \Pi_{k+1}$. Пусть $A \in \Sigma_k$, т.е $x \in A \Leftrightarrow \exists y_{1} \forall y_{2} \exists y_{3} \ldots \forall y_k : V\left(x, y_{1}, y_{2}, \ldots, y_{k}\right)=1$. Тогда $x \in A \Leftrightarrow \exists y_{0} \forall y_{1} \exists y_{2} \ldots \forall y_k : V\left(x, y_{1}, y_{2}, \ldots, y_{k}\right)=1$, где $y_0$ фиктивная переменная, и предикат $V$ ее не использует. По определению $A \in \Pi_{k+1}$. Значит $\Sigma_k \subset \Pi_{k+1}$.\\
$\Pi_k \subset \Pi_{k+1}$. Пусть $A \in \Pi_k$, т.е $x \in A \Leftrightarrow \forall y_{1} \exists y_{2} \forall y_{3} \ldots \exists y_k : V\left(x, y_{1}, y_{2}, \ldots, y_{k}\right)=1$. Тогда $x \in A \Leftrightarrow \forall y_{1} \exists y_{2} \ldots \forall y_k \exists y_{k+1} : V\left(x, y_{1}, y_{2}, \ldots, y_{k}\right)=1$, где $y_{k+1}$ фиктивная переменная, и предикат $V$ ее не использует. По определению $A \in \Pi_{k+1}$. Значит $\Pi_k \subset \Pi_{k+1}$.\\
$\Pi_k \subset \Sigma_{k+1}$. Пусть $A \in \Pi_k$, т.е $x \in A \Leftrightarrow \forall y_{1} \exists y_{2} \forall y_{3} \ldots \exists y_k : V\left(x, y_{1}, y_{2}, \ldots, y_{k}\right)=1$. Тогда $x \in A \Leftrightarrow \exists y_0 \forall y_{1} \exists y_{2} \ldots \forall y_k : V\left(x, y_{1}, y_{2}, \ldots, y_{k}\right)=1$, где $y_{0}$ фиктивная переменная, и предикат $V$ ее не использует. По определению $A \in \Sigma_{k+1}$. Значит $\Pi_k \subset \Sigma_{k+1}$.\\
Это и значит, что $\Sigma_k \cup \Pi_k \subset \Sigma_{k+1} \cap \Pi_{k+1}$.
\subsubsection*{Задача 9}
\textit{Докажите, что полиномиальная иерархия <<схлопывается>>, если существует $\mathcal{PH}c$ задача. \\
Под схлопыванием имеется в виду $\exists k: \mathcal{PH} = \Sigma_k = \Pi_k$.}\\
\\
Действительно, если язык $A \in \mathcal{NP}c$, то $A \in \mathcal{NP}$ и значит $\exists k: A \in \Sigma_k$. Т.к. $\forall B \in \mathcal{PH} \rightarrow B \leq_p A$, то $B \in \Sigma_k$. Значит $\mathcal{PH} = \Sigma_k$.\\
$\mathcal{PH} = \cup \Pi_k = \Sigma_k$, значит $\Pi_{k+n} \subseteq \Sigma_k,$ $\forall n \in \mathbf{N}$. Но $\Sigma_k \subseteq \Pi_{k+n},$  $\forall n \in \mathbf{N}$. Тогда $\Pi_k = \Sigma_k = \mathcal{PH}$.

\end{document}