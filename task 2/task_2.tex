\documentclass{article}
\usepackage[english,russian]{babel}
\usepackage[left=1cm,right=2cm,top=2cm,bottom=2cm,footskip=1cm]{geometry}
\usepackage{ amssymb }
\usepackage{amsmath} 

\title{Домашняя работа 2}
\author{Пасечник Даша}
\date{на 29.02.2019}

\usepackage{natbib}
\usepackage{graphicx}

\begin{document}

\maketitle

\section*{Задача 1}

\textit{Функция $u(M)$ равна наибольшему числу тактов работы на входных словах длины $10$, если МТ $M$ останавливается на каждом таком слове, и не определена в противном случае. Вычислима ли $u(M)$?}\\

Докажем, что $u(M)$ вычислима, т.е. построим алгоритм, который ее вычисляет.\\
Т.к. алфавит любой машины Тьюринга конечен, то конечно и число слов длины 10. Занумеруем их. Постоим таблицу (i, j), где i - номер входа из множества входов длины 10, j - номер МТ. Как заполнять ячейки таблицы? Запускаем МТ j на входе i: если она останавливается на входе i, то в ячейку (i, j) пишем число тактов, за которое она завершила работу.\\
Функцию $u(M)$ вычисляем так: идем по строке с номером М и вычисляем значения ячеек (i, M), затем проходим по строке еще раз и ищем максимум. Элементов в строках конечное число, т.к. конечно число входов длины 10, поэтому:
\begin{itemize}
	\item если M останавливается на всех входных словах длины 10, то максимум будет найден за конечное время,
	\item иначе работа МТ М на каком-то из слов не завершится и значение $u(M)$ окажется неопределено.
\end{itemize} 
Построили алгоритм вычисления $u(M)$, следовательно функция вычислима.

\section*{Задача 2}

\textit{Разрешим ли язык $L$, состоящий из всех описаний МТ, у которых есть недостижимое состояние (не достигается ни при каком входе)?}\\

Про язык $L_{\emptyset}$ известно, что он неперечислим (см задачу 3) и неразрешим.\\
Тогда показать сводимость $L_{\emptyset} \leq_m L$ -- значит доказать неразрешимость $L$.\\
Рассмотрим вспомогательную МТ $M_\$$ с одним состоянием, которая печатает $\$$ на ленту и останавливается. Кроме того рассмотрим функцию $g(M)$, получающую на вход МТ и выдающую эквивалентную ей МТ без недостижимых состояний. Приведем алгоритм работы $g$:
\begin{itemize}
	\item[1] Поместим начальное состояние в множество достижимых.
	\item[2] Проходим по таблице переходов (она конечна) и добавляем в множество достижимых состояний каждое такое, в которое есть переход из состояния уже находящегося в множестве достижимых.
	\item[3] Повторяем 2. пока множество достижимых состояний не перестанет меняться (это произойдет по крайней мере тогда, когда в множество достижимых попадут все состояния МТ - их конечное число, следовательно за конечное время).
	\item[4] Очевидно, состояния, которые не попали в множество достижимых - не достигаются))) Так что удалим их.
\end{itemize}

Т.к. недостижимые состояния не влияют на работу МТ ни на каком входе, то $g(M) \in L_{\emptyset} \Leftrightarrow M \in L_{\emptyset}$. Тогда рассмотрим функцию $f(M) = g(M) \circ M_\$$.\\ 
Если $M \in L_{\emptyset}$, т.е. $M$ не останавливается ни на каком входе, то $g(M)$ - аналогично. Значит и $f(M)$ не останавливается ни на каком входе и состояние $M_\$$ оказываются для нее недостижимым. Получили, что $f(M) \in L$\\
Если $M \notin L_{\emptyset}$, т.е. существует слово $\omega$, на котором останавливается $M$, то $g(M)$, а следовательно и $f(M)$ так же останавливаются на нем. \\
Все состояния $g(M)$ достижимы по построению, а состояние $M_\$$ достигается по крайней мере на слове $\omega$. Т.е. $f(M) \notin L$.\\
Таким образом показали, что:\\
$\left \{\begin{array}{ccc}  M \in L_{\emptyset} & \implies & f(M) \in L \\ M \not \in L_{\emptyset} & \implies & f(M) \not \in L \end{array}\right.$.\\
Т.е. $L_{\emptyset} \leq_m L$. Значит $L$ - неразрешим.

\section*{Задача 3}

\textit{Перечислим ли язык $L_\emptyset$ состоящий из всех описаний МТ, которые не останавливаются ни на каком входе?}\\

Рассмотрим сначала дополнение к $L_\emptyset$: $L_1 = L \setminus L_\emptyset$ - язык всех машин Тьюринга, которые останавливаются по крайней мере на одном входе ($L$ - язык всех машин Тьюринга). Легко показать, что существует вычислимая функция $R(x, y): \Sigma^*\times\Sigma^* \rightarrow \{0, 1\}$: $x\in L_1 \iff \exists y \in \Sigma^*: R(x, y) = 1$. $R(x, y)$ будет запускать МТ $x$ на входе $y$ и выдавать 1, если МТ $x$ останавливается на входе $y$, 0 - иначе. Действительно, если $x \in L_1$, то по построению $L_1$ найдется $y$ такой, что $R(x, y) = 1$. Если $x \notin L_1$, то $x \in L_\emptyset$, а значит  $\nexists y: R(x, y) = 1$. Существование $R(x, y)$ равносильно перечислимости $L_1$.\\\\
Теперь покажем сводимость $L_{stop} \leqslant_m L_1$.\\
Т.е. нужно найти такую функцию $f$, что $\left \{\begin{array}{ccc} (M, \omega) \in L_{stop} & \implies & f((M, \omega)) \in L_1 \\ (M, \omega) \not \in L_{stop} & \implies & f((M, \omega)) \not \in L_1 \end{array}\right.$\\
Рассмотрим $f((M, \omega)) = M_w\circ M$, где $M_w$ -- вспомогательная МТ, которая считывает слово $\omega$ с ленты и переводит головку в начало слова. Заметим, что МТ $ M_{w_0}\circ M_0$ не завершает свою работу на всех словах кроме $\omega_0$, т.к. МТ $M_{w_0}$ не завершает свою работу на всех словах кроме $\omega_0$. На слове $\omega_0$ МТ $M_{w_0}$ завершает свою работу и передает управление МТ $M_0$. Головка при этом находится в начале слова $\omega_0$, т.е. $M_0$ начинает работу на входе $\omega_0$. Таким образом, если $(M_0, \omega_0) \in L_{stop}$, то $f((M_0, \omega_0))$ останавливается на входе $\omega_0$, т.е. $f((M, \omega)) \in L_1$. Иначе, $f((M_0, \omega_0))$ не останавливается ни на каком входе, т.е. $f((M, \omega)) \notin L_1$.\\
Получили, что $f$ -- искомая функция, а значит $L_{stop} \leqslant_m L_1$.\\
Тогда из неразрешимости $L_{stop}$ следует неразрешимость $L_1$, а из неразрешимости и перечислимости $L_1$ по теореме Поста -- неперечислимость $L \setminus L_1$, т.е. $L_\emptyset$.\\
Ответ: $L_\emptyset$ - неперечислим.

\section*{Задача 4}

\textit{Показать, что любой перечислимый язык сводится к $L_{stop}$.}\\

Рассмотрим перечислимый язык $ A $. Для него $ \exists g $ -- вычислимая функция, область определения которой -- множество $ A $. Тогда существует МТ $ M_g $, вычисляющая эту функцию, т.е. такая, что она останавливается на словах из $ A $, и не останавливается иначе.\\
Т.к. $L_{stop} =_m L_{stop, \emptyset}$, то сводимость к $L_{stop, \emptyset}$ равносильна сводимости к $L_{stop}$. Покажем, что существует $f(x)$:\\
$\left \{\begin{array}{ccc} x \in A & \implies & f(x) \in L_{stop, \emptyset} \\ x \not \in A & \implies & f(x) \not \in L_{stop, \emptyset} \end{array}\right.$\\
Рассмотрим вспомогательную МТ: $ M_x $, которая печатает слово $x$ и возвращает головку в начало слова. Тогда функция $ f(x) = M_x \circ M_g $ -- искомая. Действительно, 
если $ x\in A $, то $M_x \circ M_g$ остановится на пустом входе, т.к. $M_x$ останавливается на любом входе, а  $M_x$ на входах из $A$. Т.е.  $f(x) \in L_{stop, \emptyset} $.\\
Если $ x\notin A$, то $M_x \circ M_g$ не остановится на пустом входе, т.к. $M_g$ не останавливается на входе $x$. Т.е. $ f(x)\notin L_{stop} $.
Таким образом, по определению $m$-сводимости, любой перечислимый язык сводится к $ L_{stop, \emptyset} $, т.е. и к $L_{stop}$.

\section*{Задача 5}

\textit{Верно ли, что все непустые коперечислимые языки $m$-сводятся друг к другу?}\\

Докажем о/п: пусть верно, что все непустые коперечислимые языки $m$-сводятся друг к другу.\\
Конечные языки разрешимы (ссылаемся на семинар), а следовательно и перечислимы. Тогда по теореме Поста они не могут быть не коперечислимыми.\\
Язык $L_\emptyset$ коперечеслим и неперечислим (см. задачу 3), следовательно неразрешим. \\
Тогда для произвольного конечного языка $A$ и языка $L_\emptyset$ выполняется, например, $A \leq_m L_{\emptyset}$. Тогда из разрешимости $A$ следует разрешимость $L_\emptyset$. Получили противоречие.\\
Ответ: не верно.

\section*{Задача 6}

\textit{Функция Трудолюбия Радо (busy beaver function) определяется, как максимальное количество единиц, которые может напечатать МТ с $n$ состояниями перед остановкой. 
\begin{itemize}
\item Всюду ли эта функция определена?
\item (Доп) Вычислима ли эта функция?
\end{itemize}}

а) Останавливающихся МТ с $ n $ состояниями конечное число, т.к. таблицы переходов конечны. Время их работы конечно, т.е. всегда можем найти максимум по числу напечатанных единиц среди них. Это и будет значение $ \Sigma(n) $.\\
Ответ: $ \Sigma(n) $ всюду определена.\\

б) Докажем, что функция трудолюбия Радо $ \Sigma(n) $ невычислима.\\
Пусть $ f(n) $ -- произвольная вычислимая функция. Рассмотрим функцию $ g(x) = max[f(2x+2),f(2x+3)] + 1$.  Вычислимось $ g $ следует из вычислимости $ f $. \\
Тогда существует $ M $ -- МТ, которая вычисляет функцию $g$, пусть у нее $ k $ состояний. Рассмотрим также служебную МТ: $ M^* $, которая будет писать $ x + 1 $ единиц на ленте (для этого нужно $ x + 2 $ состояния) и останавливаться. МТ $ N_x = M^* \circ M $ участвует в соревновании среди МТ с $ x + k + 2 $ состояниями, значит $ g(x) \leq \Sigma(x + k + 2)$. Тогда выполняется $ f(2x + 2) < \Sigma(x + k + 2)$ и $f(2x + 3) < \Sigma(x + k + 2)$, а для $ x \geq k $: $ f(2x + 2) < \Sigma(2x + 2) $ и $ f(2x + 3) < \Sigma(2x + 2) < \Sigma(2x + 3)$.\\
То есть при больших $ n $ (чётных и нечётных) выполняется $ f(n) < \Sigma(n) $. Получили, что $ \Sigma(n) $ растёт быстрее любой всюду определённой вычислимой функции, поэтому не является вычислимой.\\
Ответ: $\Sigma(n)$ невычислима.


\section*{Задача 7}
\textit{Постройте биекции: 
\begin{itemize}
\item $(0, 1) \rightarrow (0, +\infty)$
\item $[0, 1] \rightarrow [0, 1)$
\item $[0, 1] \rightarrow [0, 1]^2$
\item $2^\mathbb{N} \rightarrow [0, 1]$
\end{itemize}}

Решение:
\begin{itemize}
\item $(0, 1) \rightarrow (0, +\infty)$\\

Зададим биективной функцией $f(x) = 1/x$. Если область определения - $(0, +\infty)$ (здесь функция взаимнооднозначна), то область значений: $(0, 1)$.\\

\item $[0, 1] \rightarrow [0, 1)$\\

Построим такую функцию: $f(1)=\frac{1}{2}$, $f(\frac{1}{2})=\frac{1}{3}$, и далее для всех чисел вида $x = 1/n: f\left(\frac{1}{n}\right)=\frac{1}{n+1}$. Для остальных чисел с $[0, 1]$: $f(n) = n$. Функция взаимнооднозначна, область значений - $[0, 1)$.\\

\item $[0, 1] \rightarrow [0, 1]^2$\\

Покажем, что отрезок $[0, 1]$ равномощен множеству бесконечных последовательностей из нулей и единиц. Каждое число $x \in [0, 1]$ можем записать в виде бесконечной двоичной дроби: первый знак после запятой равен 0, если $x$ лежит в левой половине отрезка $[0, 1]$, и равен 1, если в правой. Чтобы определить следующий бит, нужно поделить выбранную половину снова пополам. Если $x$ лежит в левой половине, то следующая цифра 0, а если в правой, то 1. И так далее: чтобы определить очередной знак, нужно поделить текущий отрезок пополам и посмотреть, в какую половину попадает x.\\
Однако сейчас одному числу может соответствовать 2 бесконечные двоичные последовательности. Это происходит, когда точка попадает на границу очередного отрезка. Тогда мы можем относить её как к левой, так и к правой половине. В результате, например, последовательности $0,1001111 \cdots$ и $0,101000\cdots$ соответствуют одному и тому же числу.
Чтобы сделать взаимнооднозначное соответствие исключим последовательности, в которых начиная с некоторого момента все цифры равны 1 (кроме одной: $ 0,1111 \ldots $ - соответствует 1). Таких последовательностей счётное множество, так что их добавление не меняет мощности множества. В итоге получили биекцию между отрезоком $[0, 1]$ и множеством бесконечных последовательностей нулей и единиц.\\

Тогда квадрат $[0, 1]\times [0, 1]$ равномощен множеству упорядоченных пар таких последовательностей (пара соответсвующая точке $(x,y)$ - пара из последовательности соответстувующей $x$ и последовательности соответствующей $y$).\\
Установим отображение между бесконечными последовательностями нулей и единиц и парами таких последовательностей: паре $ (a_0,a_1,a_2 \ldots, b_0,b_1,b_2 \ldots) $ ставим в соответствие последовательность $ a_0,b_0,a_1,b_1,a_2,b_2 \ldots $. Это отображение взаимно однозначное (обратное к нему выделяет из последовательности отдельно чётные и отдельно нечётные члены).\\
Т.е. построили искомую биекцию.\\

\item $2^\mathbb{N} \rightarrow [0, 1]$\\

Представим множество всех подмножеств множества натуральных чисел в виде бесконечных последовательностей нулей и единиц (1 -- берём элемент в множество, 0 -- не берём). Биекцию между множеством бесконечных последовательностей нулей и единиц и $[0, 1]$ доказали в прошлом пункте.
\end{itemize}
\end{document}