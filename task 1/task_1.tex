\documentclass{article}
\usepackage[english,russian]{babel}

\title{Домашняя работа 1}
\author{Пасечник Даша}
\date{на 15.02.2019}

\usepackage{natbib}
\usepackage{graphicx}

\begin{document}

\maketitle

\section*{Задача 1}

Построим двухленточную машину Тьюринга - Т. Сначала Т проходит по слову, написанному на первой ленте, и переписывает его на вторую. Затем головка первой ленты возращается к началу слова. Далее головка первой ленты идет вправо, а головка второй ленты влево, и на каждом шаге переход осуществляется только если обе головки указывают на одинаковые символы. Таким образом, если слово является палиндромом, то Т проидет по слову сначала до конца, иначе - остановится внутри слова. \\Приведем формальное описание Т:\\ \\
$T = \left(A, Q, Q_{f}, q_{0}, \delta, \Lambda\right)$\\
$Q = \lbrace q_0, q_1, q_2, q_3\rbrace$\\
$A = \lbrace \lambda, a, b \rbrace$\\
$Q_f = \lbrace q_3\rbrace$\\
\\
$\delta:$\\
$$\delta\left(q_{0}, a, \Lambda\right)=\left(q_{0}, a, a,+1,+1\right)$$
$$\delta\left(q_{0}, b, \Lambda\right)=\left(q_{0}, b, b, +1,+1\right)$$
$$\delta\left(q_{0}, \Lambda, \Lambda\right)=\left(q_{1}, \Lambda, \Lambda,-1,0\right)$$
$$\delta\left(q_{1}, a, \Lambda\right)=\left(q_{1}, a, \Lambda,-1,0\right)$$
$$\delta\left(q_{1}, b, \Lambda\right)=\left(q_{1}, b, \Lambda,-1,0\right)$$
$$\delta\left(q_{1}, \Lambda, \Lambda\right)=\left(q_{2}, \Lambda, \Lambda, +1,-1\right)$$
$$\delta\left(q_{2}, a, a\right)=\left(q_{2}, a, a,+1,-1\right)$$
$$\delta\left(q_{2}, b, b\right)=\left(q_{2}, b, b,+1,-1\right)$$
$$\delta\left(q_{2}, \Lambda, \Lambda\right)=\left(q_{3}, \Lambda,\Lambda,0,0\right)$$

\section*{Задача 2}

Доказать, что следующие определения перечислимого множества $X \subset {N}$ эквивалентны:
\begin{itemize}
\item Существует алгоритм, печатающий все элементы множества (в любом порядке и со сколь угодно большими паузами между элементами).
\item Множество является областью определения некоторой вычислимой функции.
\item Множество является областью значений некоторой вычислимой функции.
\end{itemize}


$(1\Rightarrow3)$ Если существует алгоритм, печатающий все элементы множества, то множество является областью значений некоторой вычислимой функции.\\
Построим функцию $f(n)$, область значений которой - X.\\
На входе n запустим алгоритм, печатающий элементы Х, и будем считать число напечатанных элементов. Когда будет напечатан n-ый элемент - выдадим его в качестве значения функции $f$ на входе n. Множество значений функции f - X.

$(3 \Rightarrow 2)$ Если множество является областью значений некоторой вычислимой функции, то множество является областью определения некоторой вычислимой функции. \\
Пусть функция $f(n)$ имеет область значений X. Построим функцию $g(x)$ такую, что её область определения - Х. На входе х будем перебирать все натуральные числа и подавать на вход f(n). Если $f(n) = x$, то $g(x) = 1$, иначе - $g(x)$ не определена в х. \\Получим, что X - область определения g.

$(2 \Rightarrow 3)$ Если множество является областью значений некоторой вычислимой функции, то множество является областью определения некоторой вычислимой функции.\\
Пусть g(x) - функция, область значений которой - X. Переопределим g(x) так, что для каждого х из Х $g(x)=x$. Теперь Х - область значений некоторой вычислимой функции.

$(3 \Rightarrow 1)$ Если множество является областью определения некоторой вычислимой функции, то существует алгоритм, печатающий все элементы множества.\\
Пусть функция $f(n)$ имеет область значений X. Для каждого натурального числа, если функция f(n) определена, то печатаем результат. Это и будет алгоритм, печатающий все элементы множества Х и только их.
\\\\
Эквивалентность доказана.

\section*{Задача 3}

Дан массив из n элементов, на которых определено отношение равенства (например, речь может идти о массиве картинок или музыкальных записей). Постройте алгоритм, который в «потоковом режиме обработки данных» определяет, есть ли в массиве элемент, повторяющийся больше $n/2$ раз. Считается, что в вашем распоряжении есть память объемом $O(log n)$ битов.

Введем две дополнительных переменных $ans$ и  $counter$: в переменной $ans$ в каждый момент времени находится элемент массива предположительно встречающийся больше $n/2$ раз, $counter$ - это счетчик.

Алгоритм:

1. В $ans$ кладем первый элемент массива, $counter = 0$.

2. При первом проходе по массиву на каждом шаге выполняем следующие действия:
\begin{itemize}
    \item Если $counter = 0$, записываем текущий элемент массива в $ans$,  $counter = 1$.
    
    \item Если $counter != 0$, сравниваем $ans$ с текущим элементом массива: если совпадают, то $counter += 1$, иначе $counter -= 1$.
\end{itemize}
3. Если искомый элемент существует, то после прохода по массиву он будет лежать в $ans$. Поэтому обнуляем $counter$ и делаем второй проход по массиву: сравниваем каждый элемент с $ans$. Если текущий элемент совпадает с $ans$, $counter += 1$. 

4. Если $counter > n/2$, выводим $ans$.

Сложность данного алгоритма $O(n)$, а требуемая дополнительная память — $O(1)$.

\section*{Задача 4}
На вход подается описания n событий в формате (s,f) — время начала и время окончания. Требуется составить расписание для человека,который хочет принять участие в максимальном количестве событий.

Алгоритм 1:

Выберем событие кратчайшей длительности, добавим его в расписание, исключим из рассмотрения события, пересекающиеся с выбранным. Продолжим делать то же самое далее.

Контрпример:
    Пусть есть события (1, 5), (6, 10), (4, 7). Алгоритм 1 выберет событие (4, 7), как кратчайшее и исключит из рассмотрения (1, 5) и (6, 10), как пересекающиеся с ним. Больше событий не осталось, следовательно, результат работы алгоритма: 1. Очевидно, что существует расписание, при котором можно посетить 2 события: (1, 5) и (6, 10), т.к. они не пересекаются. Вывод: алгоритм не оптимальный.

Алгоритм 2:

Выберем событие, наступающее раньше всех, добавим его в расписание, исключим из рассмотрения события, пересекающиеся с выбранным. Продолжим делать то же самое далее.

Контрпример:
    Пусть есть события (1, 10), (2, 3), (4, 5). Алгоритм 2 выберет событие (1, 10), как наступающее раньше всех и исключит из рассмотрения (2, 3) и (4, 5), как пересекающиеся с ним. Больше событий не осталось, следовательно, результат работы алгоритма: 1. Очевидно, что существует расписание, при котором можно посетить 2 события: (2, 3) и (4, 5), т.к. они не пересекаются. Вывод: алгоритм не оптимальный.

Алгоритм 3:

Выберем событие, завершающееся раньше всех, добавим его в расписание, исключим из рассмотрения события, пересекающиеся с выбранным. Продолжим делать то же самое далее.


Докажем от противного, что Алгоритм 3 работает корректно:

Среди всех примеров, где алгоритм работает неоптимально, выберем тот, который содержит минимальное число событий. Посмотрим на первое событие, выбранное алгоритмом: это событие, которое раньше всех заканчивается (Обозначим его С1). Пусть оно не входит в оптимальное решение. Значит, первое событие в оптимальном решении (обозначим его С2) заканчивается позже (не раньше) события С1. Тогда мы можем поменять С2 на С1 в оптимальном решении, т.к. это не изменит число событий в оптимальном решении и не приведет к пересечениию между событиями. Теперь решение, составленное алгоритмом 3, и оптимальное начинаются с одного и того же события. Отбросим его. Получим пример, в котором решение, составленное Алгоритмом 3 все еще не оптимальное, а рассматриваемых событий меньше, чем в предыдущем. Получили противоречие с тем, что изначально рассматриваемый пример был минимальным.

Т.о. Алгоритм 3 всегда работает корректно, так что выбираем его.
Предварительная сортировка по времени конца события ${O}(n\log{}n)$, проход по массиву: ${O}(n)$. 
Следовательно, сложность алгоритма: ${O}(n\log{}n)$.

\section*{Задача 5}
Найдите явное аналитическое выражение для производящей
функции чисел $BR_{4n+2}$ правильных скобочных последовательностей
длины 4n + 2.

Обозначим $BR_{4n+2}$ через $T_{2n+1}$. Для чисел Каталана известно реккурентное соотношение: $$T_n = T_0T_{n-1} + T_1T_{n-2}+ ... + T_{n-1}T_0$$ $$T_0 = 1$$

Производящая функция для чисел Каталана длины $4n+2$, т.е. содержащих $2n+1$ открывающуюся скобку имеет вид: $$A(x) = T_1+T_3x+ T_5x^2+...+T_{2n+1}x^n+...$$
Рассмотрим также производящую функцию для чисел Каталана длины $4n$: $$B(x) = T_0+T_2x+ T_4x^2+...+T_{2n}x^n+...$$

Распишем их произведение: 
$$AB = (T_1+T_3x+T_5x^2+...)(T_0+T_2x+T_4x^2+...)$$
$$AB = T_1T_0+(T_3T_0+T_1T_2)x+(T_1T_4+T_2T_3+T_5T_0)x^2+...$$
Заметим, что:
$$T_2 = T_1T_0+T_0T_1 => T_1T_0 = T_2/2$$
Аналогично:
$$T_3T_0+T_1T_2 = T_4/2$$
$$T_1T_4+T_2T_3+T_5T_0 = T_6 / 2$$
$$...$$
Тогда:
$$AB = \frac{T_2}{2}+\frac{T_4}{2}x+\frac{T_6}{2}x^2+...+\frac{T_{2n}}{2}x^{n-1+...}$$
Домножим обе части равества на 2х, тогда справа окажется B(x) без первого члена $T_0$:
$$2xAB = T_2x+T_4x+...+T_{2n}x^n+...$$
$$2xAB = B-T_0$$
Т.к. $T_0 = 1$:
\begin{equation}
    2xAB = B-1
\end{equation}
Теперь распишем $B^2(x)$:
$$B^2 = (T_0+T_2x+T_4x^2+...)(T_0+T_2x+T_4x^2+...)$$
$$B^2 = T_0^2+(T_0T_2+T_2T_0)x+(T_0T_4+T_2T_2+T_4T_0)x^2+...$$
Заметим, что:
$$T_0^2 = T_1$$
$$T_0T_2+T_2T_0 = T_3-T_1^2$$
$$T_0T_4+T_2T_2+T_4T_0 = T_5-(T_1T_3+T_3T_1)$$
$$...$$
Тогда:
$$B^{2}=T_{1}+T_{3} x-T_{1}^{2} x+T_{5} x^{2}-(T_{1}T_{3}+T_{3}T_{1}) x^{2}+...$$
Распишем $A^2$:
$$A^2 = (T_1+T_3x+T_5x^2+...)(T_1+T_3x+T_5x^2+...)$$
$$A^2 = T_1^2+(T_1T_3+T_3T_1)x+(T_1T_5+T_3T_3+T_5T_1)x^2+...$$
Тогда выражение для $B^2$ представимо в виде:
\begin{equation}
    B^2 = A-A^2x
\end{equation}
Объединим (1) и (2) в систему. Напомним, что $A$ - производящая функция для чисел Каталана длины $4n+2$, $B$ - производящая функция для чисел Каталана длины $4n$. Нужно найти А, но если искать напрямую получим уравнение 4 степени. Так что вместо этого найдем сначала В, затем А из (2).
$$B^{2}=A-A^{2}x$$
$$B-2xAB=1  \Rightarrow A=\frac{B-1}{B \cdot 2 x}$$

$$B^{2}=\frac{B-1}{2 x B}-\frac{(B-1)^{2}}{4 B^{2} x^{2}} x$$
$$ 4xB^{4}=(B-1)(2 B-B+1)$$
$$ 4xB^{4}=(B-1)(B+1)$$
$$ 4xB^{4}=B^{2}-1$$
Решаем квадратное уравнение:
$$B_{12}^{2}=\frac{1 \pm \sqrt{1-16 \cdot x}}{8 x}$$
Какое из решений выбрать?? Заметим, что $B(x=0)=T_0=1$. Тогда:
    $$8xB^{2}=1 \pm \sqrt{1-16 x}$$
    $$0=1 \pm \sqrt{1}$$
Чтобы сохранялось равенство нужен знак -:
    $$B^{2}=\frac{1-\sqrt{1-16 x}}{8 x}$$
    
Решим (2) как квадратное уравнение отностительно А:
$$A=\frac{1-\sqrt{1-4 x B^{2}}}{2 x}$$
Знак - взят по тем же причинам.

Теперь подставляем выражение для $B^2$ и получаем ответ:
$$A=\frac{1-\sqrt{1-\frac{1-\sqrt{1-16 x}}{2}}}{2 x}$$

\section*{Задача 6}
Оцените трудоемкость рекурсивного алгоритма, разбивающего исходную задачу размера $n$ на три задачи размером $\lceil\frac{n}{\sqrt{3}}\rceil-5$, используя для этого $10\frac{n^3}{\log n}$ операций.


Строим реккуренту:
$$ T(n)=3 T\left(\left[\frac{n}{\sqrt{3}}\right]-5\right)+\Theta\left(10 \frac{n^{3}}{\log n}\right)$$
Асимптотически нам неважны константы и целое округление:
$$T(n)=3 T\left(\frac{n}{\sqrt{3}}\right)+\Theta\left( \frac{n^{3}}{\log n}\right)$$
Найдем зависимость $\Theta(...)$ от глубины рекурсии:
$$ \frac{n^{3}}{\log n} \Rightarrow 3 \cdot \frac{\frac{n^{3}}{(\sqrt{3})^{3}}}{\log \left(\frac{n}{\sqrt{3}}\right)}=\frac{\frac{n^{3}}{\sqrt{3}}}{\log \left(\frac{n}{\sqrt{3}}\right)}\Rightarrow 9 \cdot \frac{\frac{n^{3}}{(\sqrt{3})^{3}}}{\log \left(\frac{n}{(\sqrt{3})^{2}}\right)}=\frac{\frac{n^{3}}{\sqrt{3}}}{\log \left(\frac{n}{(\sqrt{3})^{2}}\right)}$$
Заметим, что k-ый элемент имеет вид:
$$ \frac{\frac{n^{3}}{(\sqrt{3})^{k}}}{\log \left(\frac{n}{(\sqrt{3})^{k}}\right)}$$
Глубина рекурсии $\log n$, т.к. $T(n)$ выражается через $T(n/const)$. Поэтому трудоемкость алгоритма выражается суммой:
$$T(n)=\sum\limits_{k=0}^{\log (n)} \frac{\frac{n^{3}}{(\sqrt{3})^{k}}}{\log \left(\frac{n}{(\sqrt{3})^{k}}\right)}$$
Т.к. все слагаемые положительны, оценим её снизу первым членом:
$$T(n) = \Omega(\frac{n^{3}}{\log n})$$
Докажем, что сверху выполняется та же оценка по индукции по глубине рекурсии.

Б.И.: $k = 0$, тогда сумма представляет собой первое слагаемое и равна $\frac{n^{3}}{\log n}$. Доказано.

Ш.И.: Пусть доказано для $k < m$, докажем для $k = m$:

По предположению индукции верно:
$$\exists C_1>0: \exists N_1>0: \forall n \geq N_1 \mapsto \sum_{k=0}^{m-1} \frac{\frac{n^{3}}{(\sqrt{3})^{k}}}{\log \left(\frac{n}{(\sqrt{3})^k}\right)} \leq C_1 \frac{n^{3}}{\log n}$$
Преобразуем слагаемое с номером m:
$$ \frac{\frac{1}{(\sqrt{3})^{m}}}{\log \left(\frac{n}{(\sqrt{3})^m}\right)} = \frac{1}{(\sqrt{3})^{m}(\log n-m \cdot \log (\sqrt{3}))}
$$
Теперь оценим его:
$$
\frac{1}{(\sqrt{3})^{m}(\log n-m \cdot \log (\sqrt{3}))} \leq C \frac{1}{\log n}
$$
$$
\log n \leq C(\sqrt{3})^{m}(\log n-m \log \sqrt{3})
$$
$$
m \log \sqrt{3} \leq\left(C \cdot(\sqrt{3})^{m}-1\right) \log n
$$
$$
m \leq \frac{\left(C(\sqrt{3})^{m}-1\right)}{\log \sqrt{3}} \log n
$$
Для сколь угодно большого m найдутся достаточно большие С и N такие, что неравенство выполняется. Тогда слагаемое с номером m можно оценить сверху $C \frac{1}{\log n}$.
Сложим это неравенство с неравенством из предположения индукции: для  $С_2 > C_1 + C$ и $N_2 = max(N_1, N)$ выполняется:
$$\exists C_2>0: \exists N_2>0: \forall n \geq N_2 \mapsto \sum_{k=0}^{m} \frac{\frac{n^{3}}{(\sqrt{3})^{k}}}{\log \left(\frac{n}{(\sqrt{3})^k}\right)} \leq C_2 \frac{n^{3}}{\log n}$$

Шаг индукции доказан.

Таким образом, $O(\frac{n^3}{\log n})$ оценка сверху, следовательно ответ:
$$T(n) = \Theta\left(\frac{n^3}{\log n}\right)$$

\end{document}
